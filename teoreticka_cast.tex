% !TeX encoding = UTF-8
% !TeX spellcheck = sk_SK
\documentclass[]{tukediphc}
%% -----------------------------------------------------------------
%% tento subor ma kodovanie utf-8
%%
%% na kompilaciu pouzivajte format pdflatex 
%%
%% V pripade problemov kontaktujte Jána Bušu st. (jan.busa@tuke.sk)
%%
%% November 2015
%% -----------------------------------------------------------------
%%
%\usepackage[dvips]{graphicx}
%\DeclareGraphicsExtensions{.eps}
\usepackage[pdftex]{graphicx}
\DeclareGraphicsExtensions{.pdf,.png,.jpg,.mps}
\graphicspath{{figures/}} % priecinok na obrazky
%%


%\usepackage[utf8]{inputenc}  % je v cls-subore
%\usepackage[T1]{fontenc}  % je v cls-subore
\usepackage{lmodern,textcase}
\usepackage[slovak]{babel}
\def\refname{Zoznam použitej literatúry}
\usepackage{latexsym}
\usepackage{dcolumn} % zarovnanie cisiel v tabulke podla des. ciarky
\usepackage{hhline}
\usepackage{amsmath,amsfonts,amssymb}
\usepackage{nicefrac} % pekne zlomky
\usepackage{upgreek} % napr. $\upmu\mathrm{m}$ pre mikrometer ...
\usepackage[final]{showkeys}%color%notref%notcite%final
\usepackage[slovak,noprefix]{nomencl}
\makeglossary % prikaz na vytvorenie suboru .glo


% Pouzit v pripade velkeho poctu subsection v tableofcontents
%\makeatletter
%\renewcommand*\l@subsection{\@dottedtocline{2}{1.5em}{3.5em}}
%\newcommand*\l@subsection{\@dottedtocline{2}{1.5em}{2.3em}}
%\newcommand*\l@subsubsection{\@dottedtocline{3}{3.8em}{3.2em}}
%\makeatother


%\def\thefigure{\Roman{section}.\arabic{figure}}

%\usepackage{parskip}% 'zhusti' polozky obsahu
%% Cislovane citovanie
\usepackage[numbers]{natbib}
%%
%% Citovanie podľa mena autora a roku
%\usepackage{natbib} \citestyle{chicago}
% -----------------------------------------------------------------
%% tlač !!!
\usepackage[pdftex,unicode=true,bookmarksnumbered=true,
bookmarksopen=true,pdfmenubar=true,pdfview=Fit,linktocpage=true,
pageanchor=true,bookmarkstype=toc,pdfpagemode=UseOutlines,
pdfstartpage=1]{hyperref}
\usepackage{graphicx}
\hypersetup{%
pdfcreator={pdfcsLaTeX},
pdfkeywords={ekonometria, diplomová práca},
pdftitle={Návrh softvérovej podpory pre výučbu predmetu Ekonometria},
pdfauthor={Dávid Semják},
pdfsubject={Bakalárska práca}
} 

\dippraca{Diplomová práca}

\nazov{Návrh softvérovej podpory pre výučbu predmetu Ekonometria}

\podnazov{}
\jazyk{Slovenský}
% anglicky nazov
\title{Software proposal for the Econometrics course teaching}
\autor{Bc. Dávid Semják}
\veduciprace{prof. Ing. Vladimír Gazda, PhD.}
\titul{Bc.}
\univerzita{Technická univerzita v~Košiciach}
\fakulta{Ekonomická fakulta}
\skratkafakulty{Ekf}
\katedra{Katedra financií}
\skratkakatedry{KF}
\odbor{Ekonómia a manažment}
\specializacia{Financie, bankovníctvo a investovanie}
\abstrakt{Cieľom práce je zrozumiteľne vysvetliť všetky kľúčové myšlienky, potrebné k pochopeniu podstaty ekonometrie. Ekonometria oplýva množstvom techník, teoretická časť preto obsahuje selekciu poznatkov, ktoré autor považuje za kľúčové, pre pochopenie tejto vednej disciplíny. Dôraz sa kladie na nenáročnú interpretáciu, pretože práca je cielená na začiatočníkov v danom odbore. Praktická časť je zameraná na pomoc pri zvládnutí praktických častí výučby Ekonometrie, spolu s vysvetľovaním fungovania a podstaty používaných techník, doplnená o ďalšie štatistické koncepty, ktoré sa študentom zídu, avšak na cvičeniach nie je čas im venovať dostatočnú pozornosť.}
\klucoveslova{ekonometria, diplomová práca, regresia, heteroskedasticita, autokorelácia, multikolinearita, estimátor, očakávaná hodnota}
\abstrakte{Goal of the thesis is to explain all key ideas, needed for understanding basis of econometrics, in a comprehensible way. Econometrics is full of useful techniques, theoretical part of the thesis is therefore selection of techniques, that are considered crucial for understanding foundation of econometrics. Thesis put great emphasis on ease of presentation, due to target audience, that is mainly composed of newcomers to this discipline. Practical part is aimed to help readers with undergoing practical part of schooling, together with further explaining of key statistical concepts, that are useful, but due to lack of time, aren't targeted enough during classes.}
\keywords{econometrics, diploma, regression, heteroscedasticity, autocorrelation, multicollinearity, estimator, expected value}
\datumodovzdania{30.~4.~2021}
\datumobhajoby{24.~5.~2021}
\mesto{Košice}
\pocetstran{\pageref{page:posledna}}
\kategoria{Záverečná práca}

\begin{document}
\renewcommand{\figurename}{Obrázok}	
\renewcommand\theHfigure{\theHsection.\arabic{figure}}
\renewcommand\theHtable{\theHsection.\arabic{table}}
\bibliographystyle{dcu}

\prvastrana

\titulnastrana

\abstraktsk % abstrakt v SK 

\abstrakteng % abstrakt v ENG

\kabstrakt % koniec abstraktov, nova strana


\includegraphics[width = \textwidth]{obrazky/0.png}


\newpage

\cestnevyhlasenie


\podakovanie
Ďakujem vedúcemu diplomovej práce prof. Ing. Vladimír Gazda, PhD. za jeho cenné poznatky, rady a pripomienky, ktorými mi bol nápomocný pri tvorbe tejto práce.
\kpodakovania

\kpredhovoru

\thispagestyle{empty}
\tableofcontents
\newpage

\thispagestyle{empty}

{	\makeatletter
	\renewcommand{\l@figure}{\@dottedtocline{1}{1.5em}{3.5em}}
	\makeatother
	\listoffigures}

%\addcontentsline{toc}{section}{\numberline{}Zoznam obrázkov}
%\listoffigures


\newpage

\thispagestyle{empty}
%\addcontentsline{toc}{section}{\numberline{}Zoznam tabuliek}
\listoftables
\newpage

\thispagestyle{empty}
%\addcontentsline{toc}{section}{\numberline{}Zoznam symbolov a
%skratiek}

\newpage


\include{Úvod}
\section{Úvod}

bla b la bla bla

\section{formulacia}
%
\include{analyza}

\section{Náhodný výber a výberové rozdelenie}

Mnoho štatistických a ekonometrických postupov pracuje s priemermi, alebo váženými priemermi vzorky. Vysvetlenie rozdelenia priemeru, je dôležitý krok vpred, smerom k porozumeniu fungovania ekonometrických postupov. Predstavíme si základné koncepty náhodného výberu, a rozdelení priemerov. Začneme náhodným výberom. Proces náhodného výberu znamená, že náhodne vyberieme vzorku z väčšej populácie. Keďže sme vzorku vybrali náhodne, priemer vzorky môžeme považovať za náhodnú premennú. Pretože je priemer vzorky náhodnou premennou, má rozdelenie pravdepodobnosti, nazývané ako výberové rozdelenie. Ďalej si povieme o niektorých vlastnostiach výberového rozdelenia priemeru vzorky. 

\subsection{Náhodný výber}

Náhodných výberov existuje viacero. My začneme s jednoduchým náhodným výberom. Predstavme si študenta, ktorý by sa chcel stať štatistikom, tak sa rozhodne zaznamenávať, ako dlho mu trvá cesta do školy v rôzne dni. Náhodne vyberie dni, v ktoré bude čas merať. Keďže boli dni vybrané náhodne, ak budeme vedieť čas cesty v jeden deň, nepovie nám to nič o tom, ako dlho bude trvať cesta v iný deň. To, že dni meraní boli vybrané náhodne znamená, že namerané časy sú nezávisle rozdelené náhodné premenné. Opísaná situácia je príkladom najjednoduchšej formy výberu, používanej v štatistike. Pri jednoduchom náhodnom výbere sa zvolí $n$ objektov, ktoré sú náhodne vybrané z populácie. Každý člen populácie, má rovnakú šancu byť zvolený. Pozorovania $n$, si vo vzorke označíme ako $Y1, ..., Y_n$ kde $Y_1$ znázorňuje prvé pozorovanie, $Y_2$ druhé pozorovanie, a tak ďalej. V uvedenom príklade, je $Y_1$ prvý, z náhodne vybraných dní, kedy si študent zaznamenal čas, a $Y_i$ predstavuje čas v $í-tom$ náhodne vybranom dni. 
Keďže členovia populácie, ktorí boli vybraní do vzorky, boli vybraní náhodne, hodnoty pozorovaní $Y1, ..., Y_n$ sú samy náhodné. Ak by boli vybraní iní členovia z populácie, ich hodnoty $Y_i$ sa budú líšiť. Chceme tým povedať, že proces náhodného výberu znamená, že s $Y1, ..., Y_n$ môžeme zaobchádzať ako s náhodnými premennými. Pred výberom vzorky, môžu na seba hodnoty $Y1, ..., Y_n$ prevziať mnoho rozličných hodnôt. Avšak, pri výbere z populácie, sú zaznamenané konkrétne hodnoty. Pretože $Y1, ..., Y_n$ sú náhodne vybrané z rovnakej populácie, marginálne rozdelenie $Y_i$, je rovnaké pre každé $i = 1, ..., n$, teda $Y1, ..., Y_n$ sú považované za identicky distribuované. V súlade s jednoduchým náhodným výberom, poznanie hodnoty premennej $Y_1$, nám neposkytne žiadnu informáciou o $Y_2$, teda združené rozdelenie $Y_2$ daný $Y_1$, je rovnaké ako marginálne rozdelenie $Y_2$. Inými slovami, pri jednoduchom náhodnom výbere, je $Y_1$ distribuované nezávisle od $Y2, ..., Y_n$. 
Ak sú $Y1, ..., Y_n$ vybrané z rovnakého rozdelenia a sú nezávisle rozdelené, hovoríme, že sú nezávisle a identicky distribuované\footnote{independently and identically distributed}. [1]

\subsection{Výberové rozdelenie priemernej hodnoty vzorky}

Priemerná hodnota vzorky, $\overline{Y}$, $n$ pozorovaní $Y1, ..., Y_n$ je:
\begin{equation}
\overline{Y} = \frac{1}{n}(Y_1 + Y_2 + ... + Y_n)=\frac{1}{n}\sum_{i=1}^{n}Y_i.
\end{equation}

Podstatným konceptom je, že proces výberu náhodnej vzorky, urobí z priemernej hodnoty vzorky, $\overline{Y}$, náhodnú premennú. Pretože bola vzorka vybraná náhodne, hodnota každého $Y_i$, je náhodná. Pretože $Y1, ..., Y_n$ sú náhodné, ich priemer je takisto náhodný. Ak by sme vybrali inú vzorku, hodnoty pozorovaní, aj ich priemer, by sa líšili. Hodnota $\overline{Y}$ sa líši od vzorky k vzorke. Nezabúdajme, že stále máme na mysli náhodne vybrané vzorky. Napríklad, predpokladajme, že náš študent náhodne vybral päť dní, počas ktorých meral čas cesty do školy. Následne vyrátal priemer týchto piatich dní, a dostal istú hodnotu. Ak by vybral iných päť dní, počas ktorých by meral čas cesty do školy, a znova vyrátal priemernú hodnotu trvania cesty do školy, tento priemer by sa líšil od prvého vyrátaného priemeru. Keďže je $\overline{Y}$ náhodná, má vlastné rozdelenie pravdepodobnosti. Rozdelenie $\overline{Y}$ nazývame výberové rozdelenie $\overline{Y}$, pretože je to rozdelenie pravdepodobnosti, súvisiace s možnými hodnotami $\overline{Y}$, ktoré mohli byť vypočítané pre odlišné vhodné vzorky $Y1, ..., Y_n$. Rozdelenie výberu priemernej hodnoty, a váženej priemernej hodnoty, hrá v štatistike a ekonometrii dôležitú úlohu. Preto sa oboznámime s výpočtom priemeru a rozptylu výberového rozdelenia, s ohľadom na hlavné podmienky rozdelenia populácie $Y$. 

\subsection{Priemer a rozptyl $\overline{Y}$ }

Predpokladajme, že pozorovania $Y1, ..., Y_n$ sú nezávisle a identicky distribuované. $\mu_Y$ a $\sigma^2_Y$ označujú priemer a rozptyl $Y_i$. Keďže sú pozorovania $n.i.d.$, priemer je rovnaký pre všetky $i = 1, ..., n$, a to isté sa dá povedať o rozptyle. Overíme to vzťahom:  

\begin{equation}
E(X+Y)=E(X)+E(Y)=\mu_x+\mu_Y    
\end{equation}

Ak $n = 2$, priemer súčtu $Y_1 + Y_2$ bude podľa rovnice $3.1$: $E(Y_1+Y_2) = \mu_Y + \mu_Y = 2\mu_Y$. Teda priemer priemernej hodnoty vzorky je $E[\frac{1}{2}(Y_1 + Y_2)]= \frac{1}{2}2\mu_Y=\mu_Y$. Ak to zovšeobecníme, tak:

\begin{equation}
E(\overline{Y}) = \frac{1}{n}\sum_{i=1}^{n}E(Y_i) = \mu_Y.
\end{equation}

Rozptyl $\overline{Y}$ zistíme pomocou vzťahu $var(X + Y) = var(X) + var(Y) = \sigma^2_X + \sigma^2_Y$ (ak X a Y sú nezávislé). Pre $n = 2$ sa rozptyl rovná $var(Y1 + Y2) = 2\sigma^2_Y$, teda $var(\overline{Y}) =  \frac{1}{2} \sigma^2_Y$. Teda:

\begin{equation}

 var(\overline{Y}) = var \frac{1}{n} \sum_{i=1}^{n}Y_i = var(\frac{Y_1 + Y_2 +, ..., Y_n}{n})\footnote{Pri násobení konštantou, rozptyl násobíme danou konštantou umocnenou na druhú mocninu.}  = 
 

\end{equation}








\include{jadroprace}
%
\include{zaver}
%
\include{literatura}
%
\section*{Zoznam pr\'iloh}
\addcontentsline{toc}{section}{\numberline{}Zoznam pr\'iloh}
\thispagestyle{empty}

\begin{description}
	\item[Príloha A] CD médium -- záverečná práca v~elektronickej podobe.
\end{description}
%
\include{prilohaa}
%
\include{prilohab}
%
\include{prilohac}
%
% zivotopis autora
\newpage
\phantomsection
\protect\label{page:posledna}


\end{document}
%%