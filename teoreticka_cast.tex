% !TeX encoding = UTF-8
% !TeX spellcheck = sk_SK
\documentclass[]{tukediphc}
%% -----------------------------------------------------------------
%% tento subor ma kodovanie utf-8
%%
%% na kompilaciu pouzivajte format pdflatex 
%%
%% V pripade problemov kontaktujte Jána Bušu st. (jan.busa@tuke.sk)
%%
%% November 2015
%% -----------------------------------------------------------------
%%
%\usepackage[dvips]{graphicx}
%\DeclareGraphicsExtensions{.eps}
\usepackage[pdftex]{graphicx}
\DeclareGraphicsExtensions{.pdf,.png,.jpg,.mps}
\graphicspath{{figures/}} % priecinok na obrazky
%%


%\usepackage[utf8]{inputenc}  % je v cls-subore
%\usepackage[T1]{fontenc}  % je v cls-subore
\usepackage{lmodern,textcase}
\usepackage[slovak]{babel}
\def\refname{Zoznam použitej literatúry}
\usepackage{latexsym}
\usepackage{dcolumn} % zarovnanie cisiel v tabulke podla des. ciarky
\usepackage{hhline}
\usepackage{amsmath,amsfonts,amssymb}
\usepackage{nicefrac} % pekne zlomky
\usepackage{upgreek} % napr. $\upmu\mathrm{m}$ pre mikrometer ...
\usepackage[final]{showkeys}%color%notref%notcite%final
\usepackage[slovak,noprefix]{nomencl}
\makeglossary % prikaz na vytvorenie suboru .glo


% Pouzit v pripade velkeho poctu subsection v tableofcontents
%\makeatletter
%\renewcommand*\l@subsection{\@dottedtocline{2}{1.5em}{3.5em}}
%\newcommand*\l@subsection{\@dottedtocline{2}{1.5em}{2.3em}}
%\newcommand*\l@subsubsection{\@dottedtocline{3}{3.8em}{3.2em}}
%\makeatother


%\def\thefigure{\Roman{section}.\arabic{figure}}

%\usepackage{parskip}% 'zhusti' polozky obsahu
%% Cislovane citovanie
\usepackage[numbers]{natbib}
%%
%% Citovanie podľa mena autora a roku
%\usepackage{natbib} \citestyle{chicago}
% -----------------------------------------------------------------
%% tlač !!!
\usepackage[pdftex,unicode=true,bookmarksnumbered=true,
bookmarksopen=true,pdfmenubar=true,pdfview=Fit,linktocpage=true,
pageanchor=true,bookmarkstype=toc,pdfpagemode=UseOutlines,
pdfstartpage=1]{hyperref}
\hypersetup{%
pdfcreator={pdfcsLaTeX},
pdfkeywords={ekonometria, diplomová práca},
pdftitle={Návrh softvérovej podpory pre výučbu predmetu Ekonometria},
pdfauthor={Dávid Semják},
pdfsubject={Bakalárska práca}
} 

\dippraca{Diplomová práca}

\nazov{Návrh softvérovej podpory pre výučbu predmetu Ekonometria}

\podnazov{}
\jazyk{Slovenský}
% anglicky nazov
\title{Software proposal for the Econometrics course teaching}
\autor{Bc. Dávid Semják}
\veduciprace{prof. Ing. Vladimír Gazda, PhD.}
\titul{Bc.}
\univerzita{Technická univerzita v~Košiciach}
\fakulta{Ekonomická fakulta}
\skratkafakulty{Ekf}
\katedra{Katedra financií}
\skratkakatedry{KF}
\odbor{Ekonómia a manažment}
\specializacia{Financie, bankovníctvo a investovanie}
\abstrakt{Cieľom práce je zrozumiteľne vysvetliť všetky kľúčové myšlienky, potrebné k pochopeniu podstaty ekonometrie. Ekonometria oplýva množstvom techník, teoretická časť preto obsahuje selekciu poznatkov, ktoré autor považuje za kľúčové, pre pochopenie tejto vednej disciplíny. Dôraz sa kladie na nenáročnú interpretáciu, pretože práca je cielená na začiatočníkov v danom odbore. Praktická časť je zameraná na pomoc pri zvládnutí praktických častí výučby Ekonometrie, spolu s vysvetľovaním fungovania a podstaty používaných techník, doplnená o ďalšie štatistické koncepty, ktoré sa študentom zídu, avšak na cvičeniach nie je čas im venovať dostatočnú pozornosť.}
\klucoveslova{ekonometria, diplomová práca, regresia, heteroskedasticita, autokorelácia, multikolinearita, estimátor, očakávaná hodnota}
\abstrakte{Goal of the thesis is to explain all key ideas, needed for understanding basis of econometrics, in a comprehensible way. Econometrics is full of useful techniques, theoretical part of the thesis is therefore selection of these techniques, that are considered crucial for understanding foundation of econometrics. Thesis put great emphasis on ease of presentation, due to target audience, that is mainly composed of newcomers to this discipline. Practical part is aimed to help readers with undergoing practical part of schooling, together with further explaining of key statistical concepts, that are useful, but due to lack of time aren't targeted enough during classes.}
\keywords{econometrics, diploma, regression, heteroscedasticity, autocorrelation, multicollinearity, estimator, expected value}
\datumodovzdania{30.~4.~2021}
\datumobhajoby{24.~5.~2021}
\mesto{Košice}
\pocetstran{\pageref{page:posledna}}
\kategoria{Záverečná práca}

\begin{document}
\renewcommand{\figurename}{Obrázok}	
\renewcommand\theHfigure{\theHsection.\arabic{figure}}
\renewcommand\theHtable{\theHsection.\arabic{table}}
\bibliographystyle{dcu}

\prvastrana

\titulnastrana

\abstraktsk % abstrakt v SK 

\abstrakteng % abstrakt v ENG

\kabstrakt % koniec abstraktov, nova strana

% Na tomto mieste bude vložené zadanie diplomovej práce
\zadanieprace

\cestnevyhlasenie


\podakovanie
Na tomto mieste môže byť vyjadrenie poďakovania napr. vedúcemu
diplomovej práce, resp. konzultantom, za pripomienky a~odbornú pomoc
pri vypracovaní diplomovej práce.

Na tomto mieste môže byť vyjadrenie poďakovania napr. vedúcemu
diplomovej práce, respektíve konzultantom, za pripomienky a~odbornú
pomoc pri vypracovaní diplomovej práce.

Na tomto mieste môže byť vyjadrenie poďakovania napr. vedúcemu
diplomovej práce alebo konzultantom za pripomienky a~odbornú pomoc pri
vypracovaní diplomovej práce.
\kpodakovania

\predhovor
Predhovor je povinnou náležitosťou záverečnej práce, pozri
\citep{gonda}. V~predhovore autor uvedie základné charakteristiky
svojej záverečnej práce a~okolnosti jej vzniku. Vysvetlí dôvody, ktoré
ho viedli k~voľbe témy, cieľ a~účel práce a~stručne informuje
o~hlavných metódach, ktoré pri spracovaní záverečnej práce použil.
\kpredhovoru

\thispagestyle{empty}
\tableofcontents
\newpage

\thispagestyle{empty}

{	\makeatletter
	\renewcommand{\l@figure}{\@dottedtocline{1}{1.5em}{3.5em}}
	\makeatother
	\listoffigures}

%\addcontentsline{toc}{section}{\numberline{}Zoznam obrázkov}
%\listoffigures


\newpage

\thispagestyle{empty}
%\addcontentsline{toc}{section}{\numberline{}Zoznam tabuliek}
\listoftables
\newpage

\thispagestyle{empty}
%\addcontentsline{toc}{section}{\numberline{}Zoznam symbolov a
%skratiek}
\printglossary
\newpage

%\addcontentsline{toc}{section}{\numberline{}Slovník termínov}
\slovnikterminov

\begin{description}
	\item[Dizertácia] je rozsiahla vedecká rozprava, v~ktorej sa na
základe vedeckého výskumu a~s~použitím (využitím) bohatého dokladového
materiálu  ako i~vedeckých metód rieši zložitý odborný problém.
	\item[Font] je súbor, obsahujúci predpisy na zobrazenie textu
v~danom písme, napr. na tlačiarni. To čo vidíme je písmo; font je súbor
a~nevidíme ho.
	\item[Kritika] je odborne vyhrotený, prísny pohľad na hodnotenú
vec. Medzi recenziou a kritikou je taký pomer ako medzi diskusiou a
polemikou. Pri kritike treba prísnosť chápať v~tom zmysle, že sa
v~nej okrem iného navrhuje, ako hodnotené dielo skvalitniť.
	\item[Meter (m)] je vzdialenosť, ktorú svetlo vo vákuu prejde
za čas. interval~$\nicefrac{1}{299\,792\,458}$ sekundy.
	\item[Písmom] rozumieme vlastný vzhľad znakov.
	\item[Problém] termín používaný vo všeobecnom zmysle vo vzťahu
k~akejkoľvek duševnej aktivite, ktorá má nejaký rozoznateľný cieľ.
Samotný cieľ nemusí byť v~dohľadne. Problémy možno charakterizovať
tromi rozmermi -- oblasťou, obtiažnosťou a veľkosťou.
	\item[Proces] je postupnosť či rad časovo usporiadaných
udalostí tak, že každá predchádzajúca udalosť sa zúčastňuje na
determinácii nasledujúcej udalosti.
\end{description}

\kslovnikterminov
%
\include{Úvod}
\section{Úvod}

bla b la bla bla

% !TeX encoding = UTF-8
% !TeX spellcheck = sk_SK
% !TeX root=tukedip.tex
\section{Formulácia úlohy}
Na písanie textu záverečnej práce sa používajú štýly uvedené v~tejto
šablóne (Nadpis záverečnej práce, Podnadpis záverečnej práce, Text
záverečnej práce [riadkovanie 1.5, Latin Modern %Times New Roman 
12] a~ďalšie podľa
potreby). Text záverečnej práce musí obsahovať kapitolu s~formuláciou
úlohy resp. úloh riešených v~rámci záverečnej práce. V~tejto časti
autor rozvedie spôsob, akým budú riešené úlohy a~tézy formulované
v~zadaní práce. Taktiež uvedie prehľad podmienok riešenia.
%
\include{analyza}

%

\include{jadroprace}
%
\include{zaver}
%
\include{literatura}
%
\section*{Zoznam pr\'iloh}
\addcontentsline{toc}{section}{\numberline{}Zoznam pr\'iloh}
\thispagestyle{empty}

\begin{description}
	\item[Príloha A] CD médium -- záverečná práca v~elektronickej podobe.
\end{description}
%
\include{prilohaa}
%
\include{prilohab}
%
\include{prilohac}
%
% zivotopis autora
\newpage
\phantomsection
\protect\label{page:posledna}


\end{document}
%%