% Options for packages loaded elsewhere
\PassOptionsToPackage{unicode}{hyperref}
\PassOptionsToPackage{hyphens}{url}
%
\documentclass[
]{article}
\usepackage{amsmath,amssymb}
\usepackage{lmodern}
\usepackage{ifxetex,ifluatex}
\ifnum 0\ifxetex 1\fi\ifluatex 1\fi=0 % if pdftex
  \usepackage[T1]{fontenc}
  \usepackage[utf8]{inputenc}
  \usepackage{textcomp} % provide euro and other symbols
\else % if luatex or xetex
  \usepackage{unicode-math}
  \defaultfontfeatures{Scale=MatchLowercase}
  \defaultfontfeatures[\rmfamily]{Ligatures=TeX,Scale=1}
\fi
% Use upquote if available, for straight quotes in verbatim environments
\IfFileExists{upquote.sty}{\usepackage{upquote}}{}
\IfFileExists{microtype.sty}{% use microtype if available
  \usepackage[]{microtype}
  \UseMicrotypeSet[protrusion]{basicmath} % disable protrusion for tt fonts
}{}
\makeatletter
\@ifundefined{KOMAClassName}{% if non-KOMA class
  \IfFileExists{parskip.sty}{%
    \usepackage{parskip}
  }{% else
    \setlength{\parindent}{0pt}
    \setlength{\parskip}{6pt plus 2pt minus 1pt}}
}{% if KOMA class
  \KOMAoptions{parskip=half}}
\makeatother
\usepackage{xcolor}
\IfFileExists{xurl.sty}{\usepackage{xurl}}{} % add URL line breaks if available
\IfFileExists{bookmark.sty}{\usepackage{bookmark}}{\usepackage{hyperref}}
\hypersetup{
  pdftitle={Sprievodca Ekonometriou},
  hidelinks,
  pdfcreator={LaTeX via pandoc}}
\urlstyle{same} % disable monospaced font for URLs
\usepackage[margin=1in]{geometry}
\usepackage{graphicx}
\makeatletter
\def\maxwidth{\ifdim\Gin@nat@width>\linewidth\linewidth\else\Gin@nat@width\fi}
\def\maxheight{\ifdim\Gin@nat@height>\textheight\textheight\else\Gin@nat@height\fi}
\makeatother
% Scale images if necessary, so that they will not overflow the page
% margins by default, and it is still possible to overwrite the defaults
% using explicit options in \includegraphics[width, height, ...]{}
\setkeys{Gin}{width=\maxwidth,height=\maxheight,keepaspectratio}
% Set default figure placement to htbp
\makeatletter
\def\fps@figure{htbp}
\makeatother
\setlength{\emergencystretch}{3em} % prevent overfull lines
\providecommand{\tightlist}{%
  \setlength{\itemsep}{0pt}\setlength{\parskip}{0pt}}
\setcounter{secnumdepth}{-\maxdimen} % remove section numbering
\ifluatex
  \usepackage{selnolig}  % disable illegal ligatures
\fi

\title{Sprievodca Ekonometriou}
\author{}
\date{\vspace{-2.5em}}

\begin{document}
\maketitle

\hypertarget{uxfavod}{%
\section{Úvod}\label{uxfavod}}

\hypertarget{sprievodca-ekfmetriou}{%
\subsection{Sprievodca ekfmetriou}\label{sprievodca-ekfmetriou}}

Sprievodca ekonometriou má za úlohu priblížiť Vám ekonometriu, a pomôcť
Vám jej porozumieť. Sprievodcu píšem ako študent, ktorý sa ekonometriu
začal učiť sám, a sám si prešiel zdĺhavým procesom bádania a
usmerňovania. Sprievodca je zostrojený ako-tak súbežne s osnovou a
zadaniami, ktoré obdržíte na hodine. Nebudeme sa konkrétne držať
vypracovania zadaní, ale skôr princípmi, z ktorých zadania ťažia. Mnoho
študentov tento predmet nezaujíma, a zadania vypracujú okopírovaním
postupov starších spolužiakov, nuž, pochopiť ekonometriu a jej postupy
nie je vôbec jednoduché, a dokážem pochopiť, keď si študenti hľadajú
skratky. Na druhú stranu, ekonometria predstavuje skvelú vstupnú bránu
do sveta analytiky. Človek je zavalený Machine Learningom, Data Sciencom
a AI-čkom, nuž až po sfúknutí pozlátka zistí, že je to zmes matematiky,
štatistiky a počítačovej vedy (CS -- computer science). Ekonometria je
teda skvelou výhovorkou, ako oprášiť matematiku, doučiť sa štatistiku, a
naučiť sa troška programovania. R-ko sa môže zdať ako jazyk, ktorý žije
v tieni Pythonu, avšak, akoby nejedna Dominika vedela povedať, netreba
sa nechať voviesť do omylu. R-ko je najvhodnejší programovací jazyk pre
štatistikov, Google ho zahrnul do najnovších kurzov Google Analytics.
Mojou úlohou je pomôcť Vám prekonať problém, ktorý som na začiatku
svojej cesty ekonometriou vôbec nepovažoval za problém, a to množstvo
materiálov, ktoré zavalí študenta. Pomôžem Vám postupne zložiť skladačku
konceptov a teórií, na ktorých ekonometria stojí. Náročnosť
prezentovania konceptov bude prispôsobená. Nemá zmysel vysvetľovať
odvodzovanie každého estimátora. Cieľom je poskytnúť všeobecný náhľad
ekonometrie, a pomôcť Vám pochopiť, a teda príjemnejšie zvládnuť predmet
Ekonometria.

~

\textbf{Čo nás čaká:}

\begin{itemize}
\tightlist
\item
  Základy programovania v R\\
\item
  Intuitívny prehľad štatistických konceptov\\
\item
  Ekonometrické techniky
\end{itemize}

\newpage

\hypertarget{zuxe1klady-pruxe1ce-v-r}{%
\section{Základy práce v R}\label{zuxe1klady-pruxe1ce-v-r}}

\begin{quote}
Sprievodca je interaktívny, teda začneme stiahnutím a inštaláciou
\href{https://cran.r-project.org/mirrors.html}{R} a
\href{https://cran.r-project.org/mirrors.html}{RStudia}. R má samo o
sebe programovacie prostredie, avšak dnešným štandardom je používanie
intregrovaného vývojového prostredia (IDE) v podobe RStudia.
\end{quote}

Priradiť hodnotu

\hypertarget{import-uxfadajov}{%
\subsection{Import údajov}\label{import-uxfadajov}}

Table Header\\
\_\_\_\_\_\_\_\_\_\_\_ Table cell

no pisem tu novy text no pisem tu novy text no pisem tu novy text no
pisem tu novy text no pisem tu novy text no pisem tu novy text no pisem
tu novy text no pisem tu novy text no pisem tu novy text no pisem tu
novy text no pisem tu novy text no pisem tu novy text no pisem tu novy
text no pisem tu novy text no pisem tu novy text no pisem tu novy text
no pisem tu novy text no pisem tu novy text no pisem tu novy text

\textbf{test} \# Toto je testik hehehe \textbf{toot vraj spravi bold}

Skusme drbnut aj obrazok

no pisem tu novy text

\textbf{test} \# Toto je testik hehehe \textbf{toot vraj spravi bold}

Skusme drbnut aj obrazok

no pisem tu novy text

\textbf{test} \# Toto je testik hehehe \textbf{toot vraj spravi bold}

Skusme drbnut aj obrazok

no pisem tu novy text

\textbf{test} \# Toto je testik hehehe \textbf{toot vraj spravi bold}

Skusme drbnut aj obrazok

no pisem tu novy text

\textbf{test}

\end{document}
