% Options for packages loaded elsewhere
\PassOptionsToPackage{unicode}{hyperref}
\PassOptionsToPackage{hyphens}{url}
%
\documentclass[
]{article}
\usepackage{amsmath,amssymb}
\usepackage{lmodern}
\usepackage{ifxetex,ifluatex}
\ifnum 0\ifxetex 1\fi\ifluatex 1\fi=0 % if pdftex
  \usepackage[T1]{fontenc}
  \usepackage[utf8]{inputenc}
  \usepackage{textcomp} % provide euro and other symbols
\else % if luatex or xetex
  \usepackage{unicode-math}
  \defaultfontfeatures{Scale=MatchLowercase}
  \defaultfontfeatures[\rmfamily]{Ligatures=TeX,Scale=1}
\fi
% Use upquote if available, for straight quotes in verbatim environments
\IfFileExists{upquote.sty}{\usepackage{upquote}}{}
\IfFileExists{microtype.sty}{% use microtype if available
  \usepackage[]{microtype}
  \UseMicrotypeSet[protrusion]{basicmath} % disable protrusion for tt fonts
}{}
\makeatletter
\@ifundefined{KOMAClassName}{% if non-KOMA class
  \IfFileExists{parskip.sty}{%
    \usepackage{parskip}
  }{% else
    \setlength{\parindent}{0pt}
    \setlength{\parskip}{6pt plus 2pt minus 1pt}}
}{% if KOMA class
  \KOMAoptions{parskip=half}}
\makeatother
\usepackage{xcolor}
\IfFileExists{xurl.sty}{\usepackage{xurl}}{} % add URL line breaks if available
\IfFileExists{bookmark.sty}{\usepackage{bookmark}}{\usepackage{hyperref}}
\hypersetup{
  pdftitle={Praktická časť diplomovej práce},
  hidelinks,
  pdfcreator={LaTeX via pandoc}}
\urlstyle{same} % disable monospaced font for URLs
\usepackage[margin=1in]{geometry}
\usepackage{color}
\usepackage{fancyvrb}
\newcommand{\VerbBar}{|}
\newcommand{\VERB}{\Verb[commandchars=\\\{\}]}
\DefineVerbatimEnvironment{Highlighting}{Verbatim}{commandchars=\\\{\}}
% Add ',fontsize=\small' for more characters per line
\usepackage{framed}
\definecolor{shadecolor}{RGB}{248,248,248}
\newenvironment{Shaded}{\begin{snugshade}}{\end{snugshade}}
\newcommand{\AlertTok}[1]{\textcolor[rgb]{0.94,0.16,0.16}{#1}}
\newcommand{\AnnotationTok}[1]{\textcolor[rgb]{0.56,0.35,0.01}{\textbf{\textit{#1}}}}
\newcommand{\AttributeTok}[1]{\textcolor[rgb]{0.77,0.63,0.00}{#1}}
\newcommand{\BaseNTok}[1]{\textcolor[rgb]{0.00,0.00,0.81}{#1}}
\newcommand{\BuiltInTok}[1]{#1}
\newcommand{\CharTok}[1]{\textcolor[rgb]{0.31,0.60,0.02}{#1}}
\newcommand{\CommentTok}[1]{\textcolor[rgb]{0.56,0.35,0.01}{\textit{#1}}}
\newcommand{\CommentVarTok}[1]{\textcolor[rgb]{0.56,0.35,0.01}{\textbf{\textit{#1}}}}
\newcommand{\ConstantTok}[1]{\textcolor[rgb]{0.00,0.00,0.00}{#1}}
\newcommand{\ControlFlowTok}[1]{\textcolor[rgb]{0.13,0.29,0.53}{\textbf{#1}}}
\newcommand{\DataTypeTok}[1]{\textcolor[rgb]{0.13,0.29,0.53}{#1}}
\newcommand{\DecValTok}[1]{\textcolor[rgb]{0.00,0.00,0.81}{#1}}
\newcommand{\DocumentationTok}[1]{\textcolor[rgb]{0.56,0.35,0.01}{\textbf{\textit{#1}}}}
\newcommand{\ErrorTok}[1]{\textcolor[rgb]{0.64,0.00,0.00}{\textbf{#1}}}
\newcommand{\ExtensionTok}[1]{#1}
\newcommand{\FloatTok}[1]{\textcolor[rgb]{0.00,0.00,0.81}{#1}}
\newcommand{\FunctionTok}[1]{\textcolor[rgb]{0.00,0.00,0.00}{#1}}
\newcommand{\ImportTok}[1]{#1}
\newcommand{\InformationTok}[1]{\textcolor[rgb]{0.56,0.35,0.01}{\textbf{\textit{#1}}}}
\newcommand{\KeywordTok}[1]{\textcolor[rgb]{0.13,0.29,0.53}{\textbf{#1}}}
\newcommand{\NormalTok}[1]{#1}
\newcommand{\OperatorTok}[1]{\textcolor[rgb]{0.81,0.36,0.00}{\textbf{#1}}}
\newcommand{\OtherTok}[1]{\textcolor[rgb]{0.56,0.35,0.01}{#1}}
\newcommand{\PreprocessorTok}[1]{\textcolor[rgb]{0.56,0.35,0.01}{\textit{#1}}}
\newcommand{\RegionMarkerTok}[1]{#1}
\newcommand{\SpecialCharTok}[1]{\textcolor[rgb]{0.00,0.00,0.00}{#1}}
\newcommand{\SpecialStringTok}[1]{\textcolor[rgb]{0.31,0.60,0.02}{#1}}
\newcommand{\StringTok}[1]{\textcolor[rgb]{0.31,0.60,0.02}{#1}}
\newcommand{\VariableTok}[1]{\textcolor[rgb]{0.00,0.00,0.00}{#1}}
\newcommand{\VerbatimStringTok}[1]{\textcolor[rgb]{0.31,0.60,0.02}{#1}}
\newcommand{\WarningTok}[1]{\textcolor[rgb]{0.56,0.35,0.01}{\textbf{\textit{#1}}}}
\usepackage{longtable,booktabs,array}
\usepackage{calc} % for calculating minipage widths
% Correct order of tables after \paragraph or \subparagraph
\usepackage{etoolbox}
\makeatletter
\patchcmd\longtable{\par}{\if@noskipsec\mbox{}\fi\par}{}{}
\makeatother
% Allow footnotes in longtable head/foot
\IfFileExists{footnotehyper.sty}{\usepackage{footnotehyper}}{\usepackage{footnote}}
\makesavenoteenv{longtable}
\usepackage{graphicx}
\makeatletter
\def\maxwidth{\ifdim\Gin@nat@width>\linewidth\linewidth\else\Gin@nat@width\fi}
\def\maxheight{\ifdim\Gin@nat@height>\textheight\textheight\else\Gin@nat@height\fi}
\makeatother
% Scale images if necessary, so that they will not overflow the page
% margins by default, and it is still possible to overwrite the defaults
% using explicit options in \includegraphics[width, height, ...]{}
\setkeys{Gin}{width=\maxwidth,height=\maxheight,keepaspectratio}
% Set default figure placement to htbp
\makeatletter
\def\fps@figure{htbp}
\makeatother
\setlength{\emergencystretch}{3em} % prevent overfull lines
\providecommand{\tightlist}{%
  \setlength{\itemsep}{0pt}\setlength{\parskip}{0pt}}
\setcounter{secnumdepth}{-\maxdimen} % remove section numbering
\ifluatex
  \usepackage{selnolig}  % disable illegal ligatures
\fi

\title{Praktická časť diplomovej práce}
\author{}
\date{\vspace{-2.5em}}

\begin{document}
\maketitle

\hypertarget{sprievodca-ekfmetriou}{%
\section{Sprievodca ekfmetriou}\label{sprievodca-ekfmetriou}}

Sprievodca ekonometriou má za úlohu priblížiť Vám ekonometriu, a pomôcť
Vám jej porozumieť. Sprievodcu píšem ako študent, ktorý sa ekonometriu
začal učiť sám, a sám si prešiel zdĺhavým procesom bádania a
usmerňovania. Sprievodca je zostrojený ako-tak súbežne s osnovou a
zadaniami, ktoré obdržíte na hodine. Nebudeme sa konkrétne držať
vypracovania zadaní, ale skôr princípmi, z ktorých zadania ťažia. Mnoho
študentov tento predmet nezaujíma, a zadania vypracujú okopírovaním
postupov starších spolužiakov, nuž, pochopiť ekonometriu a jej postupy
nie je vôbec jednoduché, a dokážem pochopiť, keď si študenti hľadajú
skratky. Na druhú stranu, ekonometria predstavuje skvelú vstupnú bránu
do sveta analytiky. Človek je zavalený Machine Learningom, Data Sciencom
a AI-čkom, nuž až po sfúknutí pozlátka zistí, že je to zmes matematiky,
štatistiky a počítačovej vedy (CS -- computer science). Ekonometria je
teda skvelou výhovorkou, ako oprášiť matematiku, doučiť sa štatistiku, a
naučiť sa troška programovania. R-ko sa môže zdať ako jazyk, ktorý žije
v tieni Pythonu, avšak, akoby nejedna Dominika vedela povedať, netreba
sa nechať voviesť do omylu. R-ko je najvhodnejší programovací jazyk pre
štatistikov, Google ho zahrnul do najnovších kurzov Google Analytics.
Mojou úlohou je pomôcť Vám prekonať problém, ktorý som na začiatku
svojej cesty ekonometriou vôbec nepovažoval za problém, a to množstvo
materiálov, ktoré zavalí študenta. Pomôžem Vám postupne zložiť skladačku
konceptov a teórií, na ktorých ekonometria stojí. Náročnosť
prezentovania konceptov bude prispôsobená. Nemá zmysel vysvetľovať
odvodzovanie každého estimátora. Cieľom je poskytnúť všeobecný náhľad
ekonometrie, a pomôcť Vám pochopiť, a teda príjemnejšie zvládnuť predmet
Ekonometria.

~

\textbf{Čo nás čaká:}

\begin{itemize}
\tightlist
\item
  Základy programovania v R\\
\item
  Intuitívny prehľad štatistických konceptov\\
\item
  Ekonometrické techniky
\end{itemize}

\newpage

\hypertarget{zuxe1klady-programovania-v-r}{%
\section{Základy programovania v R}\label{zuxe1klady-programovania-v-r}}

\begin{quote}
Sprievodca je interaktívny, teda začneme stiahnutím a inštaláciou
\href{https://cran.r-project.org/mirrors.html}{R} a
\href{https://cran.r-project.org/mirrors.html}{RStudia}. R má samo o
sebe programovacie prostredie, avšak dnešným štandardom je používanie
intregrovaného vývojového prostredia (IDE) v podobe RStudia.
\end{quote}

\hypertarget{aritmetickuxe9-operuxe1tory}{%
\subsection{Aritmetické operátory}\label{aritmetickuxe9-operuxe1tory}}

\emph{Poďme teda rovno na vec. Začneme základnými funkciami.} R môžeme
používať ako kalkulačku, teda za pomoci klasických aritmetických
operátorov môžeme sčítať, odčítať, násobiť, deliť či umocňovať:

\begin{Shaded}
\begin{Highlighting}[]
\DecValTok{5} \SpecialCharTok{+} \DecValTok{5}
\end{Highlighting}
\end{Shaded}

\begin{verbatim}
## [1] 10
\end{verbatim}

\begin{Shaded}
\begin{Highlighting}[]
\DecValTok{5} \SpecialCharTok{{-}} \DecValTok{5}
\end{Highlighting}
\end{Shaded}

\begin{verbatim}
## [1] 0
\end{verbatim}

\begin{Shaded}
\begin{Highlighting}[]
\DecValTok{5} \SpecialCharTok{*} \DecValTok{5}
\end{Highlighting}
\end{Shaded}

\begin{verbatim}
## [1] 25
\end{verbatim}

\begin{Shaded}
\begin{Highlighting}[]
\DecValTok{5} \SpecialCharTok{/} \DecValTok{5}
\end{Highlighting}
\end{Shaded}

\begin{verbatim}
## [1] 1
\end{verbatim}

\begin{Shaded}
\begin{Highlighting}[]
\DecValTok{5}\SpecialCharTok{\^{}}\DecValTok{2}
\end{Highlighting}
\end{Shaded}

\begin{verbatim}
## [1] 25
\end{verbatim}

R-ko skrýva mnoho ďalších operátorov:

\begin{Shaded}
\begin{Highlighting}[]
\CommentTok{\# zobrazí zvyšok z delenia}
\DecValTok{5} \SpecialCharTok{\%\%} \DecValTok{2}
\end{Highlighting}
\end{Shaded}

\begin{verbatim}
## [1] 1
\end{verbatim}

My sa budeme zapodievať len tým, s čím sa na cvičeniach stretneme. Našou
úlohou nie je naučiť sa dokonalo ovládať R, ale naučiť sa používať ho v
dostatočnej miere, aby sme s ním zvládli to, čo v najbližšej dobe budeme
potrebovať.

\newpage

\hypertarget{baluxedky}{%
\subsection{Balíky}\label{baluxedky}}

Okrem základných operátorov budeme využívať aj funkcie:

\begin{Shaded}
\begin{Highlighting}[]
\FunctionTok{mean}\NormalTok{(}\DecValTok{2}\NormalTok{, }\DecValTok{4}\NormalTok{, }\DecValTok{6}\NormalTok{)}
\end{Highlighting}
\end{Shaded}

\begin{verbatim}
## [1] 2
\end{verbatim}

\begin{Shaded}
\begin{Highlighting}[]
\FunctionTok{abs}\NormalTok{(}\SpecialCharTok{{-}}\DecValTok{5}\NormalTok{)}
\end{Highlighting}
\end{Shaded}

\begin{verbatim}
## [1] 5
\end{verbatim}

\begin{Shaded}
\begin{Highlighting}[]
\FunctionTok{sqrt}\NormalTok{(}\DecValTok{8}\NormalTok{)}
\end{Highlighting}
\end{Shaded}

\begin{verbatim}
## [1] 2.828427
\end{verbatim}

Tieto funkcie sa nachádzajú v balíkoch, ktoré si môžeme predstaviť ako
také Addony. R figuruje balíkmi, ktoré sú predinštalované, a zahŕňajú
najpoužívanejšie a najzákladnejšie funkcie. Vyššie použíté funkcie sa
nachádzajú v balíku base. To, v akom balíku sa funkcia nachádza zistíte
po napísaní funkcie

\includegraphics{D:/Desktop/diplomka obrazky/1.png}

to však len zapredpokladu, že už máte balík nainštalovaný. Ak narazíte
na názov funkcie, ktorú chcete použiť, avšak nemáte nainštalovaný balík
a chcete zistiť jeho názov, buď si funkciu zadajte do Google, alebo do
konzoly napíšte:

\begin{Shaded}
\begin{Highlighting}[]
\CommentTok{\# ?meno funkcie}
\NormalTok{?base}

\CommentTok{\# ??meno funkcie}
\NormalTok{??base}

\CommentTok{\# help(menofunkcie)}
\FunctionTok{help}\NormalTok{(base)}
\end{Highlighting}
\end{Shaded}

My budeme často využívať predinštalované balíky \textbf{base} a
\textbf{stats}, avšak za pochodu si budeme inštalovať aj ďalšie balíky,
s ktorými sa na cvičeniach stretnete. Nový balík je potrebné prv
nainštalovať a potom ho načítať do prostredia.

\begin{Shaded}
\begin{Highlighting}[]
\CommentTok{\# stiahneme a našintalujeme pomocou R konzoly a funkcie install.packages()}
\CommentTok{\# ! názov balíka je citlivý na veľkosť písma}
\CommentTok{\# ! názov musí byť v úvodzovkách}

\FunctionTok{install.packages}\NormalTok{(}\StringTok{"fBasics"}\NormalTok{)}

\CommentTok{\# po nainštalovaní máme balík stiahnutý v našom PC, a tento príkaz už viac nepoužívame}
\CommentTok{\# ak však chceme funkcie z balíka použiť, musíme po zapnutí R{-}ka balík načítať príkazom}

\FunctionTok{library}\NormalTok{(fBasics)}

\CommentTok{\# ! tu už úvodzovky nie sú potrebné}
\end{Highlighting}
\end{Shaded}

\begin{quote}
\emph{Pri inštalovaní názov balíka zabalíme do úvodzoviek. Pri jeho
načítaní pomocou library() už úvodzovky nepíšeme. Na pohovor si vezmeme
oblek (úvodzovky), ale po prijatí už chodíme do práce bez obleku.}
\end{quote}

\hypertarget{objekty-a-vektory}{%
\subsection{Objekty a vektory}\label{objekty-a-vektory}}

Často chceme vyrátané výsledky znova použiť, a preto by bolo vhodné si
ich niekde uložiť. Na ukladanie a uskladnenie výsledkov slúžia objekty.
Každý objekt má meno a obsah. Meno si môžeme zadať akékoľvek, musí však:

\begin{itemize}
\tightlist
\item
  začínať malým alebo veľkým písmenom a nie číslom
\item
  obsahovať iba čísla, písmená alebo niektoré špeciálne znaky ako
  \("."\) či "\_".
\end{itemize}

\begin{quote}
\emph{Nezabúdajme, že R je case sensitive (rozlišuje veľké a malé
písmená).}
\end{quote}

Povedzme že chceme vytvoriť objekt \textbf{a} a priradiť mu hodnotu
\(2 + 2\). Na priradenie obsahu je možné použiť \("="\), avšak
štandardom je používanie \("<-"\).

\begin{quote}
\emph{Znak \textless- nie je nutné písať dvoma znakmi, používa sa na to
skratka \textbf{``ľavý Alt'' a ``-''}. Na SK klávesnici nájdeme znak
naľavo od pravého Shiftu, na EN klávesnici zvyčajne naľavo od Backspacu.
Osobne programujem s EN klávesnicou, nech som použíteľný v akomkoľvek
štáte bez potreby inštalovať SK klávesnicu.}
\end{quote}

\begin{Shaded}
\begin{Highlighting}[]
\CommentTok{\# Priradíme teda objektu "a" výsledok "2 + 2".}

\NormalTok{a }\OtherTok{\textless{}{-}} \DecValTok{2} \SpecialCharTok{+} \DecValTok{2}

\CommentTok{\# Po napísaní názvu objektu do konzoly nám konzola ukáže už len výsledok.}

\NormalTok{a}
\end{Highlighting}
\end{Shaded}

\begin{verbatim}
## [1] 4
\end{verbatim}

\begin{Shaded}
\begin{Highlighting}[]
\CommentTok{\# Nové priradenie hodnoty starému objektu prepíše starú hodnotu.}

\NormalTok{a }\OtherTok{\textless{}{-}} \DecValTok{5} \SpecialCharTok{+} \DecValTok{5}

\NormalTok{a}
\end{Highlighting}
\end{Shaded}

\begin{verbatim}
## [1] 10
\end{verbatim}

\#nainštalovat len jeden balik a potom ked prejdeme vektory tak c(viac
balikov)

\hypertarget{import-uxfadajov}{%
\subsection{Import údajov}\label{import-uxfadajov}}

Table Header

\begin{longtable}[]{@{}lr@{}}
\toprule
-\textgreater{} & First Header \\
\midrule
\endhead
Content Cell & Content Cell \\
Content Cell & Content Cell \\
\bottomrule
\end{longtable}

\end{document}
